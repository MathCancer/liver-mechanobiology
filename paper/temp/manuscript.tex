\documentclass[smallextended,natbib]{svjour3}

%%% custom 

\newcommand{\beqa}{\begin{eqnarray}}
\newcommand{\beq}{\begin{equation}}
\newcommand{\eeqa}{\end{eqnarray}}
\newcommand{\eeq}{\end{equation}}


\usepackage{amsmath,amssymb}
\usepackage{natbib}
\usepackage{nameref,hyperref}
\newcommand{\micron}{\mu\textrm{m}}
\newcommand{\norm}[1]{\left|\left|#1\right|\right|}

\usepackage{color}
\newcommand{\red}[1]{\textcolor{red}{#1}}


\bibliographystyle{plainnat}

\begin{document}

\title{Computational multicellular systems biology in cancer:}
% Multiscale biophysics simulations of the liver:}
% Multiscale biophysics of colon cancer liver metastases: }
\subtitle{Simulating the impact of parenchymal fluid flow and tissue mechanics on colon cancer metastases of the liver\footnote{Better titles appreciated}}
\dedication{Dedicated to all the papers we should have written.}

\author{Paul Macklin${}^{*,\dagger}$\thanks{${}^{*}$ Invited and corresponding author}\thanks{${}^{\dagger}$ Contributed equally to this work} \and Jessica Sparks${}^{\dagger}$ 
\and Erik Brodin 
\and Ahmadreza Ghaffarizadeh  \and Shannon M. Mumenthaler}
\institute{P. Macklin \and A. Ghaffarizadeh and S.M. Mumenthaler 
\at Lawrence J. Ellison Institute for Transformative Medicine, University of Southern California, 
2250 Alcazar St., CSC-248, Los Angeles, CA 90033, USA\\Tel.: 310-701-5785, \email{Paul.Macklin@MathCancer.org}, www: {MathCancer.org}
\and 
J. Sparks \and E. Brodin \at 
Dept. of Chemical, Paper and Biomedical Engineering, Miami Univiersity, (address), Oxford, OH ZIP, USA}

\titlerunning{Short title}
\authorrunning{Macklin et al.}

\maketitle


\begin{abstract}150-250 words
This is an awesome paper! 
This is an awesome paper! 
This is an awesome paper! 
This is an awesome paper! 
This is an awesome paper! 
This is an awesome paper! 
This is an awesome paper! 
This is an awesome paper! 
This is an awesome paper! 
This is an awesome paper! 
This is an awesome paper! 
This is an awesome paper! 
This is an awesome paper! 
This is an awesome paper! 
This is an awesome paper! 
This is an awesome paper! 
This is an awesome paper! 
This is an awesome paper! 
This is an awesome paper! 
This is an awesome paper! 
This is an awesome paper! 
This is an awesome paper! 
This is an awesome paper! 
This is an awesome paper! 
This is an awesome paper! 
This is an awesome paper! 
This is an awesome paper! 
This is an awesome paper! 
\keywords{Neat \and Cool}
\subclass{MSC codes}
\PACS{PACS codes}
\CRclass{CR codes}
\end{abstract}

\section{Introduction}
\red{Paul, Jessica, and Shannon write.}
%Colon cancer mets in liver are bad.  
%Prognosis known to be TME-dependent. 
%Impact of parenchymal mechanics and flow on growth poorly understood, as is the biology regulationg tumor-parenchyme 
%interactions at the tumor boundary.  
%New emerging experimetnal platforms allow probing these dynamics, but still difficult to measure flow dynamics at 
%small scales, in and around tumor foci. Modeling can help. 
Different models give different insights. We are working to integrate these insights. 
Continuum model helps motivate discrete model. 
Discrete model helps motivate continuum model. 
Part of a larger-scale effort to dynamically link codes, create open source framework 
for experimentally-driven multicellualr systems biology.  \red{Paul}

\begin{enumerate}
\item 
Colon cancer metastasizes to the liver. big clinical problem. bad.  \red{Jessica, Shannon}
\item 
Impact of TME unknown, but does impact prognosis \red{Jessica, Shannon}
\item 
Impact of parenchymal mechanics and flow on growth poorly understood, as is the biology regulating tumor-parenchyme interactions 
at the tumor boundary. \red{Jessica, Shannon}
\item 
New emerging platforms allow you to directly problem some dynamics, but can't measure everything. Difficult ot measure flow dynamics at 
small scales, in and aroudn foci. Modeling can help.  \red{Jessica, Shannon}
\item 
open questions: 
\begin{enumerate}
\item 
Where do mets most commonly seed? \red{Jessica, Shannon}
\item
How do they alter mechanics adn flow of the liver? \red{Jessica}
\item 
How do cancer cells impact surrounding hepatocytes? \red{Jessica, Shannon}
\item 
Tumor cells are often densely packed, but lower cell-cell adhesion? So, is relative permeability higher or lower? Impact on growth 
dynamics? \red{Jessica, Paul}
\end{enumerate}
\item 
math modeling is a way to fill in the missing pieces, generate testable hypotheses. \red{Paul, Jessica}
\item 
Give flavor of past modeling results. Note that most is tissue regeneration, toxicology, or single-scale models of flow.  
\red{Paul, Jessica}
\item 
And now the fun twist: in most multiscale investigations, discrete or smaller-scale simulations are used to generate insights for 
larger-scale continuum models. Here, we do the opposite: use a very detailed continuum model of flow, mechanics to generate insights 
to build a discrete multicellular model. \red{Paul}
\end{enumerate}


\begin{figure}\sidecaption 
% \resizebox{0.3\hsize}{!}{\includegraphics*{./figures/figure.eps}}
% \includegraphics*{./figures/figure.eps}
\caption{Overall schematic of trading insights between models.}
\label{fig:multiscale_models}
\end{figure}


% \citet{ghaffarizadeh15_bioinformatics} is a great paper \citep{ghaffarizadeh15_bioinformatics}. 

\subsection{Background biology \red{Shannon}}
Describe backround biology of liver (structurally: lobules with outflow from portal triads through parenchyma (sinusoids -- cords of heaptocytes surrounding channels, or sinusoids that are lined by thin, fenestratedendothelial cells), flowing out through central veins). Main role: metabolize toxins, act as a giant reactive blood sponge. :-) Also creates bile, but not main focus here.  
See Fig. \ref{fig:liver_schematic}. 

Anything else on the environment. General mechanics? General oxygenation? 

Then, the background of colon cancer mets. A preferred site for colon cancer metastases. Properties of most incomign cells. Things like this. 

Anything that Shannon things belongs here! 
% \cite{ghaffarizadeh15_bioinformatics} 

\begin{figure}[tbh]
Figure: overall liver geometry. 
\caption{Main organization of liver tissue} 
\label{fig:liver_schematic}
\end{figure}

\subsection{Prior modeling works \red{Jessica, Paul}}
Describe past modeling efforts, limite insights, still not enough for CRC mets. Prior modeling restricted to colon cancer mets in 
liver, and potentially relevant liver models (most of which are focused on tissue engineering, healing, toxicology) \red{Paul, Jessica}


Short literature review on prior models of liver flow, mechanics. 
Note much focused on PKPD, toxicity, strong work by Glazier group. 
Note much work on regeneration, strong work by Drasdo group. 
Others on flow. 
Little integration of these works. 
Pull from our NSF proposal. \red{Paul, Jessica}

% \section{Method \red{Jessica, Paul}}

\subsection{Overall approach \red{Paul}}
We willl investigate using two modeling approaches. First, in Section 
\ref{section:PVE} we will use the PVE model to udnerstand flow, 
mechanics across a typical lobule, and to generate more insights on 
the fluid / mechanical state of the liver environment surrounding 
small tumor foci. 

In Section \ref{section:ABM}, we will use agent-based model 
to investigate cell and tisue biomechanics, transport-limited 
growth, and tumor-parenchymal interactions on growing tumor foci. 
Work will direclty incorporate insights from PVE. 


\section{First approach: Poroviscoelastic (PVE) model \red{Jessica}}
\label{section:PVE}
Short intro. 1 paragraph at most. 

\subsection{Method}

Describe the PVE model of the liver lobule, including reference to prior papers 
\subsubsection{Starting model assumptions} 
\subsubsection{Insights from prior tumor models} 
Necrosis as fluid pressure source (necrotic cells release fluid), viable regions as fluid pressure sink (proliferating cells grow in part by 
absorbing fluid). Smaller tumors tend to be proliferative, with no necrotic regions, so assume they are pressure source. Medium tumors have some of both, so perhaps neutral. 
Large tumors may have more necrosis than proliferation, or may be in balance. Pressure sink or neutral. 

Any other assumptions here? 

\subsubsection{Parameter estimation}
\subsubsection{Numerical solution method}

\subsection{Results}

\subsection{Initial insights from the PVE model}

Describe the parameter space investigation, and what we learned. What constitutive  
relations for PhysiCell we drew from this work. (particularly, for where flow is intact, 
and mechanics and strain at tumor-parenchyme interface). 

\vfill
\pagebreak 

\section{Second approach: Agent-based (PhysiCell) model \red{Paul}}
\label{section:ABM}
Short intro. 1 pararaph at most. 

\subsection{Method}
Describe the model of several liver lobules.  Include overall geometry, 
% We will seed the liver radially from portal triads, fill liver sinusoids with parenchyme cells, 
% empty to central vein. 


\subsubsection{Agents}
Describe tumor cell agents: substrate-driven growth, apoptosis, necrosis. Cell-cell adhesion, repulsion. Assume no 
motiltiy for now. Proliferation rate: 
\beq
a
\eeq

Apoptosis rate
\beq
a
\eeq

Necrosis rate
\beq
a
\eeq

Describe parenchyma agents. Rather than model individual hepatocytes, endothelial cells, model parenchyma as a collection 
of agents, diameter 30 $\micron$. Each parenchyma agent will follow regular adhesion-repulsion mechanics. In addition, we model 
mechanica interactions with underlying ECM. Each agent $i$ is attached to ECM at $\vec{x}_{i,\textrm{ECM}}$, and experiences 
an elastic force. Let $\vec{d}_i = \vec{x}_{i,\textrm{ECM}} - \vec{x}_i $, and let $d_i = \norm{ \vec{d}_i } - R_i$ denote 
the ellastic deformation. Then we use a force of the form 
\beq
\frac{d}{dt} \vec{x}_{i}(t) =r_\textrm{ECM} d_i  \frac{ \vec{d}_i }{ \norm{\vec{d}_i}} 
% \frac{d}{dt} \vec{x}_{i}(t) =r_\textrm{ECM} \Bigl(    \norm{ \vec{x}_{i,\textrm{ECM}} - \vec{x}_i } - R_i \Bigr) 
% \Bigl(    \norm{ \vec{x}_{i,\textrm{ECM}} - \vec{x}_i } - R_i \Bigr) 
% \frac{ \vec{x}_{i,\textrm{ECM}} - \vec{x}_i }{ \norm{ \vec{x}_{i,\textrm{ECM}} - \vec{x}_i } }. 
\eeq

To model plastic reorganization, we will evolve $\vec{x}_{i,\textrm{ECM}}$ by 
\beq
\frac{d}{dt} \vec{x}_{i,\textrm{ECM}} = r_\textrm{mech} \left( \vec{x}_{i} - \vec{x}_{i,\textrm{ECM}} \right)
\eeq

We shall not model proliferation in the parenchyma, instead assuming that it begins in a state of homeostasis. 
However, each agent (representnig a small section of tissue), will ahve a probabiltiy of apoptosing. 
IN this paper, we will investigate the following sub-models 

\paragraph{Model 1:} Parenchymal agents enter apoptotic state if 

\paragraph{Model 2:} Parenchymal agents enter apoptotic state if deformation $d_i$ exceeds a threshold 

\paragraph{Model 3:} Parenchymal agents enter apoptotic state if cumulative deformation 
$\int_0^t d_i(s)\: ds$ exceeds a threshold. 


\subsubsection{Biotransport}
We will model the transport of a single growth substrate $\sigma$ (e.g., oxygen or glucose) 
in the tissue with 
\beqa
\frac{ \partial \sigma }{\partial t} & = & 
-\nabla \cdot \vec{J} - \lambda \sigma \label{eqn:transport1}  \\
\vec{J} &  = & -D \nabla \sigma + \sigma \vec{u} \label{eqn:transport2} , 
\eeqa
where $\vec{u}$ is the fluid flow field (from the PVE model), $\lambda$ is the growth substrate 
consumption rate, $D$ is diffusion coefficient, and we defer discussion of boundary conditions until 
later. 

\subsubsection{Integrating insights from the PVE model: flow}
Describe the simulations under the assumptions from the PVE model.  Outside 
tumor, model advective flow as relatively fast, so Dirichlet conditions outside tumor. 
Little flow inside tumors, especially big ones, so purely diffusive. 

Model smaller tumors as reasonble flow?  




To evalulate the relative contribution of the terms, we rewrite Equations \ref{eqn:transport1}-\ref{eqn:transport2} as 
\beq
\frac{ \partial \sigma}{\partial t} = D \nabla^2 \sigma - \vec{u} \cdot \nabla \sigma - \sigma \nabla \cdot \vec{u} - \lambda \sigma. 
\eeq
Approximating the flow field as slowly varying ($\vec{u} \approx u \vec{w}$ for some constant $u$ and a constant 
unit vector $\vec{w}$), $\nabla \cdot \vec{u} \approx 0$. 
%
%Let us first consider a simplified geometry, where a hepatic lobule is approximated as a cylinder 
%with radius $R_\textrm{lobule}$, and approximately radial flow $\vec{u} = -u \widehat{r}$ (where 
%$\vec{r}$ is distance from the center of the lobule, $\widehat{r}$ is the radial unit vector, and $u$ is constant). 
%Assuming that $\sigma$ also has a radial profile, Equations \ref{eqn:transport1}-\ref{eqn:transport2} 
%simplify to 
%\beq
%\frac{ \partial \sigma }{ \partial t}  = D \nabla^2 \sigma + u \frac{\partial \sigma}{\partial r}   - \lambda \sigma 
%\label{eqn:transport_radial}
%\eeq
If we nondimensionalize space %  ($r$) 
with scale $L$ and time % ($t$) 
with scale $\bar{t}$, then this becomes 
\beq
\frac{ \partial  \sigma}{\partial t}  =  
\left( \frac{ D \bar t}{ L^2 } \right) \nabla^2 \sigma  - \left(  \frac{u \bar{t} }{L} \right)\vec{w} \cdot \nabla \sigma  - \left( \lambda \bar{t} \right) \sigma. 
% \left( \frac{ D \bar t}{ L^2 } \right) \nabla^2 \sigma  + \left(  \frac{ u \bar{t} }{L} \right) \frac{ \partial \sigma}{\partial r} - \left( \lambda \bar{t} \right) \sigma. 
\eeq
Choosing 
\beq
% \frac{ D \bar{t} }{ L^2 } 
\frac{ u \bar{t}}{L } = 1 \textrm{ and } \lambda \bar{t} = 1 
\Longrightarrow \bar{t} = \frac{1}{\lambda} \textrm{ and } L = \frac{ u }{ \lambda }, 
\eeq
then the nondimensionalized equation becomes 
\beq
\frac{ \partial \sigma}{\partial t} = \left( \frac{ D \lambda }{u^2} \right) \nabla^2 \sigma - \vec{w} \cdot \nabla \sigma  - \sigma . 
\eeq
By prior work \citep{ghaffarizadeh15_bioinformatics}, $D \sim 10^5 \micron^2 / \textrm{min}$ and $\lambda \sim 10 \textrm{ min}^{-1}$, 
and by the PVE work in Section \ref{?} and \citep{nishii}, $u \sim 10^{-4} \textrm{m}/\textrm{sec} = 6 \times 10^3 \micron/\textrm{min}$, and so   
% (order of magnitude from $10^3 \:\micron/\textrm{min}$ to $10^4\:\micron/\textrm{min}$), 
$L \sim 600\:\micron$, and 
$\bar{t} \sim 6 \textrm{ sec}$. The diffusion term has relative order of magnitude 
\beq
\frac{ D \lambda }{ u^2}  \sim \frac{1}{36} 	\approx 0.03. 
\eeq
Thus, in unobstructed regions of liver lobule, flow is primarily advective-reactive, and we can simplify the original equation  to 
\beq
\frac{ \partial \sigma }{ \partial t} = -\nabla \cdot \left( \sigma \vec{u} \right) - \lambda \sigma. 
\eeq
Because the time scale is on the order of seconds, advective flow is at quasi-steady state in regions of unobstructed flow 
when considered on time scales of intermediate mechanics (minutes) and tumor growth (hours to days). To approximate 
quasi-steady oxygen distribution outside the tumore regions, we we ....  \\ 
\\

now do the cylindrical stuff. 

To avoid solving this equation 

We now seek a simplified solution in regions 



In Section \ref{?}, we saw that 


\subsubsection{Integrating insights from the PVE model: mechanics}




% \subsubsection{Starting model assumptions}

\subsubsection{Parameter estimation}

\subsubsection{Numerical solution method} Here's where we say it's implemented in 
BioFVM Version (?) for PDE solvers, and PhysiCell Version (?) for agent-based model. 


\subsection{Results}

\subsubsection{Model results: dynamics for small tumor foci, no parenchyme apoptosis}
Use the assumptions from small tumors. (Good flow in tumors, so no diffusion, no gradients, parenchyme cells moved). 

\subsubsection{Model results: dynamics for small tumor foci, delayed parenchyme apoptosis} 
Use the assumptions from small tumors. (Good flow in tumors, so no diffusion, no gradients, parenchyme cells moved, 
assume the apoptose if moved too far). 

\subsubsection{Model results: dynamics for larger tumor foci, delayed parenchyme apoptosis} 
Use the assumptions from larger tumors. (Poor flow in tumors, diffusion, parenchyme cells moved, 
assume the apoptose if moved too far). 

\subsubsection{Model results: dynamics for larger tumor foci, immediate parenchyme apoptosis} 
Use the assumptions from larger tumors. (Poor flow in tumors, diffusion, parenchyme cells apoptose if near tumor cells). 





\section{Discussion and future directions \red{Paul, Jessica, Shannon}}
Recap of what we've learned. Does it mesh well with curently known stuff? Does it help explain clinical stuff? 

Future directions: critical to incorporate interstitial fluid flow with tumor cell volume changes. Likewise, need to drive porosity assumptions 
from tumor cell density. More direct, dynamical linking of codes. 

Add biochemical signaling, active tissue remodeling (rather than just deformation), advective terms in BioFVM to replace 
the cylindrical far-field assumptions. 

Follow-up. 3D is possible, but more insight on 3D lobular geometry, flow necessary. etc. 


\begin{acknowledgements}
PQ grant, BCRF grant, PSOC grant, support of CAMM, others?
\end{acknowledgements}

\section{Supplementary Materials \red{Paul, Jessica}}
\begin{enumerate}
\item 
Digital cell lines for tumor cells, parenchyma 
\item 
PhysiCell simulation source code
\item 
Simulation output data
\item 
Visualization routines 
\item 
Analysis scripts 
\item 
Anything else? 
\end{enumerate}

\bibliography{references}

\end{document}

